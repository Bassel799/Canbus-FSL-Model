\begin{abstract}
The purpose of this study is to classify cyberattacks on Controller Area Network (CAN) Bus systems, which are essential for automotive communications, using Few-Shot Learning (FSL) techniques, more especially Prototypical Networks. Because they are inherently weak in security, traditional CAN systems are vulnerable to attacks like replay, flooding, and spoofing. The requirement for large labeled datasets in traditional supervised learning techniques is a major drawback considering the infrequency and complexity of real-world attack scenarios. In order to overcome this difficulty, We used an FSL methodology that includes episodic training techniques that allow for efficient generalization from a small number of labeled examples. A precise CAN Bus data segmentation method using temporal thresholds, the extraction of informative statistical features (such as standard deviation, mean, median, and message count), and the restructuring of data into tensor formats appropriate for convolutional neural networks (CNNs) were among the custom implementations. Achieving 100\% validation accuracy in a few epochs, the experimental results showed that the Prototypical Network achieved remarkable accuracy, confirming the method's practicality and efficacy in improving automotive cybersecurity.
\end{abstract}