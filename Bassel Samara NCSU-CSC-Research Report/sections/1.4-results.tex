\section{Model Training}
\subsection{Episodic Training}
All models were implemented in PyTorch for CUDA acceleration, and training was done in an episodic manner on an NVIDIA GPU platform. Stochastic gradient descent (SGD) with a momentum of 0.9 and weight decay of 5e-4 was used in the optimizer configuration. The learning rate was set at 0.01 and decreased by a factor of 10 at epochs 120 and 160. Fifteen randomly selected few-shot tasks with balanced support and query sets were included in each training epoch.

By the first epoch, the model achieved a remarkable 100\% validation accuracy on both datasets, which can be attributed to the effectiveness of the Prototypical Networks formulation and the discriminative power of the extracted statistical features. Consistent loss values below 1e-8 following the initial epochs demonstrated training stability, and despite the small episodic batch size, no overfitting was detected. The framework's suitability for real-time deployment scenarios was demonstrated by validation tasks that ran at 50–55 iterations per second.

\subsection{Performance Metrics}
Near-perfect classification performance was found across all datasets by quantitative evaluation. With inference times averaging less than 1 millisecond per task, the model achieved 100\% accuracy in all validation episodes on the Car-Hacking dataset (960 attack blocks, 4 attack types). The approach's generalizability was confirmed by similar outcomes for the SynCanAttacks dataset (436 blocks, 4 attack types). The efficiency of the ResNet12 backbone was especially noteworthy; when serialized, the entire Prototypical Networks system used less than 6KB of memory, which is a crucial benefit for hardware deployments of automotive quality.

A comparison with conventional supervised methods demonstrated the data efficiency of the FSL method. The suggested system achieved maximum accuracy during training with just 5 examples per class, whereas traditional deep learning models may need thousands of labeled examples per attack type. The need for flexible intrusion detection systems that can identify new threats with little retraining is directly met by this capability in the automotive sector.

\subsection{Conclusion}
This study showed that Few-Shot Learning with Prototypical Networks provides a workable way to classify CAN Bus attacks with perfect accuracy under strict computational limitations. A modified ResNet12 backbone optimized for few-shot automotive applications, an end-to-end training framework that converges quickly even with little labeled data, and a novel feature extraction pipeline that converts temporal CAN bus messages into spatially structured tensors were among the technical contributions. Future research might look into hybrid architectures that combine anomaly detection and few-shot learning to find previously undiscovered attack types. The system could then be ported to automotive-grade microcontrollers for in-car testing, and the temporal analysis window could be extended to assess performance on multi-stage attacks that develop over longer periods of time. These developments will help close the gap between scholarly research and automotive cybersecurity solutions that are ready for production.